\documentclass[a4paper,twoside]{tufte-handout}
\usepackage{listings}
\usepackage{amsmath}
\usepackage{amssymb}
\usepackage{booktabs}
\usepackage[T1]{fontenc}
\usepackage[utf8]{inputenc}
%% \usepackage{graphics}
\usepackage[USenglish]{babel}

\frenchspacing

\newtheorem{exercise}{Exercise}
\newtheorem{example}{Example}

\renewcommand{\theexercise}{\Alph{exercise}}

\newcommand\lbl[1]{\hspace{-1em}\emph{#1:}}

%%\lstset{basicstyle=\ttfamily,backgroundcolor=\color{white},frame=single,rulecolor=\color{gray!20},framesep=10pt,linewidth=12cm}
\lstset{basicstyle=\normalfont\ttfamily\small,frame=single,rulecolor=\color{gray!40}}

\title{Programming Deck Code Interpreter}
\author{Jesper Larsson, IT University of Copenhagen}
\date{2011-09-06}

\begin{document}
\maketitle

\begin{abstract}
  This text contains exercise material and a mandatory assignment for
  the course \emph{IT Foundations}, fall 2011, and is part of the
  syllabus.
\end{abstract}

%% ===============================================
\section{Introduction}\label{sec-intro}

\emph{Deck code} is a programming language based on the deck-of-cards
model from the previous exercise. Granted, a very small and
specialized language, but nevertheless a programming language, which
lets you try some of the basic techniques of writing a real program.

Using deck code differs from the previous english-language algorithm
descriptions in two ways. First, the syntax is more strictly defined,
and abbreviated, which makes it less natural to read, but saves a
lot of typing. Second, rather than ``running'' the algorithms manually
using paper cards, there is an \emph{interpreter} that can run the
code for you, so that you can test that it actually works. A strict
syntax is necessary for an interpreter to work.

The available operations are exactly the same as in the previous
exercise, so whatever algorithm you came up with then, you can now
test by writing them in deck code.

%% ===============================================
\section{Syntax}\label{sec-syntax}

This section defines the operations of the deck code language. You
will use it as a reference when trying deck code on your own. It may
be easier to understand if you take a look at a few of the examples in
the next section before getting into all the details.

An operation consists of an \emph{instruction}, followed by
\emph{arguments}. The instruction is one word, or several
space-separated words. The arguments are comma-separated. Deck names
and labels are sequences of letters and digits, without any spaces.

Everything is case sensitive. For instance, you can't write
\lstinline!Jump! instead of \lstinline!jump!, and there may be two
different labels called \lstinline!end! and \lstinline!End!.

%% -----------------------------------------------------------
\subsection{Read}\label{sec-input}

Defining input wasn't part of the informal-language algorithms, but
they have to be specified to the interpreter, and therefore need to be
included in the deck code.

The \emph{read} operation gets input. It has one deck name argument,
and one optional file name argument. For example, you can simply write:
%%
\begin{lstlisting}
   read in
\end{lstlisting}

This defines a deck called \emph{in}, which must not have been
previously defined, and stops to read its contents. When you get to
this line in a running deck code program you see:
%%
\begin{lstlisting}
Enter cards for in: 
\end{lstlisting}
%%
and you are expected to enter the contents of the deck bottom up, as a
comma-separated list of integer numbers. For instance, it can look
like this:
%%
\begin{lstlisting}
Enter cards for in: 3, 12, 7, 1
\end{lstlisting}
%%
After typing the number list, you hit return, and the program
continues, the deck \emph{in} now containing four cards, 3~on the
bottom and 1~on the top.

It gets tedious to have to type input every time, especially if you
are entering the same cards over and over for testing
purposes. Therefore, you may choose to read from a file instead, by
adding a comma and a filename, for example like this:
%%
\begin{lstlisting}
   read in, a.deck
\end{lstlisting}
%%
This assumes that there is a plain text file \lstinline!a.deck! in the
current working directory, containing a comma-separated list of integers.

%% -----------------------------------------------------------
\subsection{New}\label{sec-new}

You create a new, empty, deck with the \emph{new} operator, which has
a single deck name argument. Like this, for instance:
%%
\begin{lstlisting}
   new temp
\end{lstlisting}

%% -----------------------------------------------------------
\subsection{Movement}\label{sec-move}

You move cards between decks with either \emph{movetop}, which moves
the top card of one deck to another, or \emph{moveall}, which moves
all the cards as one pack, without rearranging them. Both
commands take two deck name arguments, the one to move from followed
by the one to move to. So, for instance:
%%
\begin{lstlisting}
   movetop in, temp
   moveall temp, out
\end{lstlisting}
%%
moves the top card from \emph{in} to \emph{temp}, and then puts all
the cards from \emph{temp} on top of out.

%% -----------------------------------------------------------
\subsection{Jumping}\label{sec-}

There are a number of forms of jump, each of which has a \emph{label}
as the last argument. The label must be present somewhere in the code:
as a word followed by a colon, on a line of its own. The variants of
jump are:
%%
\begin{itemize}
\item Unconditional, ``jump~\emph{label}''.
\item Conditioned on comparison of two cards,
%%
``jump if \emph{condition}, \emph{left}, \emph{right}, \emph{label}'',
%%
where \emph{condition} is one of ``less'', ``greater'', or
``equal''. The meaning is: jump to \emph{label} if and only if the top
card of the deck \emph{left} is ``\emph{condition-to/than}'' the top card
of deck \emph{right}. For instance,
%%
\begin{lstlisting}
  jump if less in, temp, start
\end{lstlisting}
%%
jumps to start if th top card of \emph{in} is less than the top card
of \emph{temp}.

\item Conditioned on whether a deck is empty, either ``jump if
  empty \emph{deck}, \emph{label}'' or ``jump if not empty \emph{deck},
  \emph{label}.
\end{itemize}

  
Here is a snippet of code with some examples of labels and jumps:
%%
\begin{lstlisting}
start:
   jump if equal in1, in2, end
   jump if not empty temp, start
end:
   jump start
\end{lstlisting}

Writing like this, with labels flushed left and the rest of the code
indented with a few spaces, is just to make the code easier to
read. The deck code interpreter does not care about extra spaces, so
they have no effect when you run the program.

%% -----------------------------------------------------------
\subsection{Output}\label{sec-output}

You output a deck simply with ``output \emph{deck}''. This prints the
name of the deck, and then the cards, from bottom to top, for
instance:
%%
\begin{lstlisting}
   output out1
   output out2
\end{lstlisting}

%% -----------------------------------------------------------
\subsection{stop}\label{sec-stop}

Stop is the only operation without any arguments. It's just the one word:
%%
\begin{lstlisting}
   stop
\end{lstlisting}

\section{Examples}\label{sec-examp}

The examples in this section are implementations of corresponding
algorithms in the previous exercise text (\emph{Algorithms in Natural
  Language}).

\begin{example}
  Split a deck of cards into two parts whose sizes are as equal as possible.
  \begin{description}
  \item[Input:] One deck of cards named \emph{in}
  \item[Output:] Two decks of cards \emph{out1} and \emph{out2},
    whose numbers of cards differ by at most one.
  \end{description}
\begin{lstlisting}
   read in
   new out1
   new out2
check:
   jump if empty in, end
   movetop in, out1
   jump if empty in, end
   movetop in, out2
   jump check
end:
   output out1
   output out2
   stop
\end{lstlisting}
\end{example}

\begin{example}
  Extract all cards greater than or equal to a given value.
  \begin{description}
  \item[Input:] One deck named \emph{limit} containing a single card,
    and one deck named \emph{in} containing any number of cards.
  \item[Output:] One deck named \emph{out}, containing all the cards
    from \emph{in} whose value is no less than the card in the
    \emph{limit} deck.
 \end{description}
\begin{lstlisting}
   read in, extractfrom.deck
   read limit
   new out
   new trash
check:
   jump if empty in, end
   jump if less in, limit, skip
   movetop in, out
   jump check
skip:
   movetop in, trash
   jump check
end:
   output out
stop
\end{lstlisting}
\end{example}

\begin{example}\label{extractsmallest}
  Extract the smallest card from a deck.
  \begin{description}
  \item[Input:] A deck \emph{in}.
  \item[Output:] A deck \emph{min}, containing containing only one
    card from \emph{in}, which has the smallest value of any of the
    cards in \emph{in}.
 \end{description}
\begin{lstlisting}
   read in, extractfrom.deck
   new min
   new trash
check:
   jump if empty in, end
   jump if empty min, move
   jump if less in, min, replace
   movetop in, trash
   jump check
replace:
   movetop min, trash
move:
   movetop in, min
   jump check
end:
   output min
   stop
\end{lstlisting}
\end{example}

\begin{example}\label{smallbottom}
  Place smallest cards at the bottom.
  \begin{description}
  \item[Input:] A deck \emph{in}.
  \item[Output:] A deck \emph{out}, that contains all the cards from
    in, but with the cards with the smallest value at the
    bottom. (Note that there can be more than one card with this
    value.) The other cards can be in any order.
 \end{description}
\begin{lstlisting}
   read in, b.deck
   new min
   new temp
   new out
check:
   jump if empty in, end
   jump if empty min, move
   jump if equal in, min, move
   jump if less in, min, replace
   movetop in, temp
   jump check
replace:
   moveall min, temp
move:
   movetop in, min
   jump check
end:
   moveall min, out
   moveall temp, out
   output out
   stop
\end{lstlisting}
\end{example}

%% ===============================================
\section{Editing and Running}\label{sec-editrun}

You create deck code files in any plain text editor (TextEdit,
NotePad, Emacs, \dots). The deck code interpreter is a program written
in the programming language Scala. To run it, you need:
%%
\begin{itemize}
\item The file \emph{decklib1.jar}, which contains the deck code
  interpreter.
\item The Scala language system.
\item The Java system, which Scala runs on top of.
\end{itemize}

Once you have Scala (and Java) installed, the file \emph{decklib1.jar}
on your CLASSPATH, and a deck code file that you saved under, say, the
name \emph{myexample.dcod}, you can run using the following command:
%%
\begin{lstlisting}
scala DeckInter myexample.dcod
\end{lstlisting}

At some stage during testing, you may find it useful to know which
lines, in which order, the interpreter executes. The \emph{-trace}
option prints each line as it is executed:
%%
\begin{lstlisting}
scala DeckInter -trace myexample.dcod
\end{lstlisting}

\end{document}