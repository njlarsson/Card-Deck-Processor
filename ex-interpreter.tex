\documentclass[a4paper,twoside]{tufte-handout}
\usepackage{listings}
\usepackage{amsmath}
\usepackage{amssymb}
\usepackage{booktabs}
\usepackage[T1]{fontenc}
\usepackage[utf8]{inputenc}
%% \usepackage{graphics}
\usepackage[USenglish]{babel}

\frenchspacing

\newtheorem{exercise}{Exercise}
\newtheorem{example}{Example}
\newtheorem{assignment}{Assignment}

\renewcommand{\theexercise}{\Alph{exercise}}

\newcommand\lbl[1]{\hspace{-1em}\emph{#1:}}

%%\lstset{basicstyle=\ttfamily,backgroundcolor=\color{white},frame=single,rulecolor=\color{gray!20},framesep=10pt,linewidth=12cm}
\lstset{basicstyle=\normalfont\ttfamily\small,frame=single,rulecolor=\color{gray!40}}

\title{Deck Code}
\author{Jesper Larsson, IT University of Copenhagen}
\date{2013-09-05}

\begin{document}
\maketitle


%% ===============================================
\section{Introduction}\label{sec-intro}

\emph{Deck code} is a programming language based on the deck-of-cards
model from the previous exercise. Granted, a very small and
specialized language, but nevertheless a programming language, which
lets you try some of the basic techniques of writing a real program.

Using deck code differs from the previous english-language algorithm
descriptions in two ways. First, the syntax is more strictly defined,
and abbreviated, which makes it less natural to read, but saves a
lot of typing. Second, rather than ``running'' the algorithms manually
using paper cards, there is an \emph{interpreter} that can run the
code for you, so that you can test that it actually works. A strict
syntax is necessary for an interpreter to work.

You find the interpreter online at
\verb$http://deck-code.appspot.com/$. It is deployed as a \emph{Google
  App Engine} application, and you use your Google account to sign
in. If you don't have a Google account, it is easy to sign up for one,
just follow Google's links.

The available operations are exactly the same as in the previous
exercise, so whatever algorithm you came up with then, you can now
test by writing them in deck code.

%% ===============================================
\section{Syntax}\label{sec-syntax}

This section defines the operations of the deck code language. You
will use it as a reference when trying deck code on your own. It may
be easier to understand if you take a look at a few of the examples in
the next section before getting into all the details.

An operation consists of an \emph{instruction}, followed by
\emph{arguments}. The instruction is one word, or several
space-separated words. The arguments are comma-separated. Deck names
and labels are sequences of letters and digits, without any spaces.

Everything is case sensitive. For instance, you can't write
\lstinline!Jump! instead of \lstinline!jump!, and there may be two
different labels called \lstinline!end! and \lstinline!End!.

%% -----------------------------------------------------------
\subsection{Deck definitions}

You have to define the decks that you use in your program. This is
most naturally done at the start. Decks can be either \emph{input}
decks, in which case you are prompted for their contents, or they can
be decks that start out empty.

Defining input wasn't part of the informal-language algorithms, but
they have to be specified to the interpreter, and therefore need to be
included in the deck code.

The \emph{deck} instruction defines a deck. It has one deck name
argument, after which the word \emph{input} should follow if this is
to be an input deck. For example, you can write:
%%
\begin{lstlisting}
   deck in input
\end{lstlisting}

This defines a deck called \emph{in}, which must not have been
previously defined. When the program is submitted, you are prompted
for its contents. When you get to
this line in a running deck code program you see:
%%
\begin{lstlisting}
in: 
\end{lstlisting}
%%
followed by an input pane, in which you are expected to enter the contents of the deck bottom up, as a
comma-separated list of integer numbers. For instance, it can look
like this:
%%
\begin{lstlisting}
in: 3, 12, 7, 1
\end{lstlisting}
%%
When you have entered the initial content for all input decks, you
click \emph{Run}, and the program starts executing.

You create an empty deck with the \emph{deck} instruction in the same
way, but omit the word \emph{input}. Like this, for instance:
%%
\begin{lstlisting}
   deck temp
\end{lstlisting}

%% -----------------------------------------------------------
\subsection{Movement}\label{sec-move}

You move cards between decks with either \emph{movetop}, which moves
the top card of one deck to another, or \emph{moveall}, which moves
all the cards as one pack, without rearranging them. Both
commands take two deck name arguments, the one to move from followed
by the one to move to. So, for instance:
%%
\begin{lstlisting}
   movetop in, temp
   moveall temp, out
\end{lstlisting}
%%
moves the top card from \emph{in} to \emph{temp}, and then puts all
the cards from \emph{temp} on top of out.

%% -----------------------------------------------------------
\subsection{Jumping}\label{sec-}

There are a number of forms of jump, each of which has a \emph{label}
as the last argument. The label must be present somewhere in the code,
as a word followed by a colon, on a line of its own. The variants of
jump are:
%%
\begin{itemize}
\item Unconditional, ``jump~\emph{label}''.
\item Conditioned on a comparison between two cards,
%%
``jump if \emph{condition}, \emph{left}, \emph{right}, \emph{label}'',
%%
where \emph{condition} is one of ``less'', ``greater'', or
``equal''. The meaning is: jump to \emph{label} if and only if the top
card of the deck \emph{left} is ``\emph{condition-to/than}'' the top card
of deck \emph{right}. For instance,
%%
\begin{lstlisting}
  jump if less in, temp, start
\end{lstlisting}
%%
jumps to \emph{start} if the top card of \emph{in} is less than the top card
of \emph{temp}.

\item Conditioned on whether a deck is empty, either ``jump if
  empty \emph{deck}, \emph{label}'' or ``jump if not empty \emph{deck},
  \emph{label}.
\end{itemize}

  
Here is a snippet of code with some examples of labels and jumps:
%%
\begin{lstlisting}
start:
   jump if equal in1, in2, end
   jump if not empty temp, start
end:
   jump start
\end{lstlisting}

Writing like this, with labels flushed left and the rest of the code
indented with a few spaces, is just to make the code easier to
read. The deck code interpreter does not care about extra spaces, so
they have no effect when you run the program.

%% -----------------------------------------------------------
\subsection{Output}\label{sec-output}

You output a deck simply with ``output \emph{deck}''. This prints the
name of the deck, and then the cards, from bottom to top, for
instance:
%%
\begin{lstlisting}
   output out1
   output out2
\end{lstlisting}

%% -----------------------------------------------------------
\subsection{stop}\label{sec-stop}

Stop is the only operation without any arguments. It's just the one word:
%%
\begin{lstlisting}
   stop
\end{lstlisting}

\section{Examples}\label{sec-examp}

The examples in this section are implementations of corresponding
algorithms in the previous exercise text (\emph{Algorithms in Natural
  Language}).

\begin{example}
  Split a deck of cards into two parts whose sizes are as equal as possible.
  \begin{description}
  \item[Input:] One deck of cards named \emph{in}
  \item[Output:] Two decks of cards \emph{out1} and \emph{out2},
    whose numbers of cards differ by at most one.
  \end{description}
\begin{lstlisting}
   deck in input
   deck out1
   deck out2
check:
   jump if empty in, end
   movetop in, out1
   jump if empty in, end
   movetop in, out2
   jump check
end:
   output out1
   output out2
   stop
\end{lstlisting}
\end{example}

\begin{example}\label{extractge}
  Extract all cards greater than or equal to a given value.
  \begin{description}
  \item[Input:] One deck named \emph{limit} containing a single card,
    and one deck named \emph{in} containing any number of cards.
  \item[Output:] One deck named \emph{out}, containing all the cards
    from \emph{in} whose value is no less than the card in the
    \emph{limit} deck.
 \end{description}
\begin{lstlisting}
   deck in input
   deck limit input
   deck out
   deck trash
check:
   jump if empty in, end
   jump if less in, limit, skip
   movetop in, out
   jump check
skip:
   movetop in, trash
   jump check
end:
   output out
stop
\end{lstlisting}
\end{example}

\begin{example}\label{extractsmallest}
  Extract the smallest card from a deck.
  \begin{description}
  \item[Input:] A deck \emph{in}.
  \item[Output:] A deck \emph{min}, containing containing only one
    card from \emph{in}, which has the smallest value of any of the
    cards in \emph{in}.
 \end{description}
\begin{lstlisting}
   deck in input
   deck min
   deck trash
check:
   jump if empty in, end
   jump if empty min, move
   jump if less in, min, replace
   movetop in, trash
   jump check
replace:
   movetop min, trash
move:
   movetop in, min
   jump check
end:
   output min
   stop
\end{lstlisting}
\end{example}

\begin{example}\label{smallbottom}
  Place smallest cards at the bottom.
  \begin{description}
  \item[Input:] A deck \emph{in}.
  \item[Output:] A deck \emph{out}, that contains all the cards from
    in, but with the cards with the smallest value at the
    bottom. (Note that there can be more than one card with this
    value.) The other cards can be in any order.
 \end{description}
\begin{lstlisting}
   deck in input
   deck min
   deck temp
   deck out
check:
   jump if empty in, end
   jump if empty min, move
   jump if equal in, min, move
   jump if less in, min, replace
   movetop in, temp
   jump check
replace:
   moveall min, temp
move:
   movetop in, min
   jump check
end:
   moveall min, out
   moveall temp, out
   output out
   stop
\end{lstlisting}
\end{example}

%% ===============================================
\section{Exercises}\label{sec-editrun}

The following exercises are the same as the ones for the natural
language exercise. But now, you can write your algorithms in deck
code, and actually verify that they work.

\begin{exercise}
  Split a deck of cards into three parts whose sizes are as equal as possible.
  \begin{description}
  \item[Input:] One deck of cards named \emph{in}
  \item[Output:] Three decks of cards \emph{out1}, \emph{out2}, and
    \emph{out3}, whose numbers of cards differ by at most one.
 \end{description}
\end{exercise}

\begin{exercise}
  Split a deck of cards into two decks. One with cards smaller than a
  given value, and one with cards greater than or equal to that value.
  \begin{description}
  \item[Input:] One deck \emph{limit} containing a single card, and one
    deck \emph{in} containing any number of cards.
  \item[Output:] One deck named \emph{smaller}, containing all the
    cards from \emph{in} whose values are less than the card in the
    \emph{limit} deck, and one deck named \emph{greater} that contains
    the other cards.
  \end{description}
\end{exercise}

\begin{exercise}
  Example\,\ref{smallbottom} above uses two temporary decks \emph{min}
  and \emph{temp}, which are not part of either input or
  output. Change the algorithm so that it only uses one temporary
  deck, \emph{min}, getting rid of \emph{temp}.
\end{exercise}

\begin{exercise}\label{sort}
  Place cards in order.
  \begin{description}
  \item[Input:] A deck \emph{in}.
  \item[Output:] A deck \emph{out}, containing all the cards from
    \emph{in} ordered with smaller-value cards below greater-value
    cards, i.e., no card should be on top of any card with a greater
    value.
  \end{description}
\end{exercise}

%% -----------------------------------------------------------
 \subsection{Hints for exercise \ref{sort}}\label{sec-hints}
 
 There are many ways of placing items in order, and you don't have to
 follow these hints if you find another way. But the simplest way to
 solve this in the card deck setting is probably to use an algorithm
 called \emph{selection sort}, which can be formulated as follows.

 Use four decks in total: \emph{in}, \emph{out}, \emph{temp}, and
 \emph{min}. The basic idea is to repeatedly find the smallest card
 from \emph{in} and move it to out. A sketch of the algorithm is:
%%

\begin{enumerate}
\item\label{firststep} While \emph{in} is not empty, repeat steps\,\ref{loop1start}--\ref{loop1end} \\
\item\label{loop1start} While \emph{in} is not empty:
  Look at the top card on \emph{in}. If it is smaller than the card on
  \emph{min}, make it the new \emph{min} card, and move any
  previous \emph{min} card to \emph{temp}. Otherwise (it is greater or
  equal), move it to \emph{temp}. (Repeat this step until \emph{in}
  is empty.)
\item Move \emph{min} to \emph{out}.
\item\label{loop1end} Move everything from \emph{temp} back to
  \emph{in}, and go back to check the condition in step\,\ref{firststep}.
\end{enumerate}

Note that this is a \emph{double loop}, a loop within a loop:
step\,\ref{loop1start} contains repetition, but is itself involved in
another repetition.

Also note that as stated, the algorithm ignores the fact that
\emph{min} starts out empty. This case has to be handled too.


\end{document}