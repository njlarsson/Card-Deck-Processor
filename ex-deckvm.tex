\documentclass[a4paper,twoside]{tufte-handout}
\usepackage{listings}
\usepackage{amsmath}
\usepackage{amssymb}
\usepackage{booktabs}
\usepackage[T1]{fontenc}
\usepackage[utf8]{inputenc}
%% \usepackage{graphics}
\usepackage[USenglish]{babel}

\frenchspacing

\newtheorem{exercise}{Exercise}
\newtheorem{example}{Example}
\newtheorem{assignment}{Assignment}

\renewcommand{\theexercise}{\Alph{exercise}}

\newcommand\lbl[1]{\hspace{-1em}\emph{#1:}}

%%\lstset{basicstyle=\ttfamily,backgroundcolor=\color{white},frame=single,rulecolor=\color{gray!20},framesep=10pt,linewidth=12cm}
\lstset{basicstyle=\normalfont\ttfamily\small,frame=single,rulecolor=\color{gray!40}}

\title{Compiling Deck Code}
\author{Jesper Larsson, IT University of Copenhagen}
\date{Exercise part I of 2012-11-16}

\begin{document}
\maketitle

%% ===============================================
\section{DeckVM Code}\label{sec-start}

This exercise illustrate translating human-readable code to machine
code  by
going back to Deck Code again. If you go to the Deck Code web site
\verb'http://deck-code.appspot.com/', click one of your scripts
(hopefully, you have some there from last time), and then \emph{submit} it,
you see a link on the next page that wasn't there before near the
bottom of the page: ``Compile \emph{yourscript} to DeckVM''
code.\marginpar{Since Deck Code is a such a primitive language, it
  could be argued that we should call the translation
  \emph{assembling} rather than \emph{compiling}, but never mind.}
Click that, and you will see a translation of your code into a
hypothetical machine language, \emph{DeckVM Code}, represented as
numbers. The following are the DeckVM instructions:

\medskip

\marginpar{\vspace{12em}The line that says \emph{currently unused} is a leftover from
  a previous version. Just pretend it's not there.}
\begin{tabular}{lll}
  \emph{mnemonic} & \multicolumn{2}{l}{\emph{machine code}} \\
\hline
  MOVETOP & $33$ & $= 32 + 1$ \\
  MOVEALL & $34$ & $= 32 + 2$ \\
  JUMP\_EMPTY & $83$ & $= 64 + 16 + 3$ \\
  JUMP\_NOT\_EMPTY & $84$ & $= 64 + 16 + 4$ \\
  JUMP\_LESS & $101$ & $= 64 + 32 + 5$ \\
  JUMP\_EQUAL & $102$ & $= 64 + 32 + 6$ \\
  JUMP & $71$ & $= 64 + 7$ \\
  OUTPUT & $24$ & $= 16 + 8$ \\
  READ & $25$ & $= 16 + 9$ \\
  (\emph{currently unused})& $58$ & $= 32 + 16 + 10$ \\
  STOP & $11$ & $=11$ \\
\end{tabular}

\medskip

There is a name (mnemonic) for each instruction, and in the second
column, a corresponding
numeric codeword. The numeric values may seem crazy at first: why not
just number them $1,2,3,\dots$ or something? If you look in the end of
each row, you can see that actually the instructions \emph{are}
numbered that way, but have some additional terms added, $16$, $32$,
and $64$. This is to make the numbers easier to interpret.

\begin{exercise}
  Why would this make the numbers easier to interpret? What do the
  instructions with the $32$ term have in common, for instance?
\end{exercise}

\begin{exercise}
  Why are the terms $16$, $32$, and $64$ chosen, and not some other
  numbers? What do these three numbers have in common? What is the
  effect in the binary representation?
\end{exercise}

Let's go back to an old example and see how it gets translated. This
is the \emph{extract greater or equal} program:

\begin{lstlisting}
   deck in input
   deck limit input
   deck out
   deck trash
check:
   jump if empty in, end
   jump if less in, limit, skip
   movetop in, out
   jump check
skip:
   movetop in, trash
   jump check
end:
   output out
stop
\end{lstlisting}

When we compile this, we get:

\begin{lstlisting}
Decks:      4
1:          2    105 110
2:          5    108 105 109 105 116
3:          3    111 117 116
4:          5    116 114 97 115 104

Inputs:     2    1 2

Program:    20
0:          83 1 17
3:          101 1 2 12
7:          33 1 3
10:         71 0
12:         33 1 4
15:         71 0
17:         24 3
19:         11
\end{lstlisting}

The numbers on the right side are the actual machine code. Everything
to the left of a colon is just there to make it easier for you to
read. Let's go through it all.

\begin{itemize}
\item The first line says how many decks the program uses (4).
\item Then there is a section with lines numbered from 1 to the number of decks,
  which lists the names of the decks. The first number after the colon
  is the length of name, and the rest are character code values. If you look
  in your favorite Ascii
  %% 
  \marginpar{Actually, the values are Unicode in
    \textsc{utf}-16 format but for values less than 128, this is the
    same as Ascii.}
  %%
  code table, you see that 105 is~'i' and 110
  is~'n', so \verb'105 110' means \emph{in}, which is the name of deck
  number~1.
\item The third section is about which decks are inputs. It starts
  with the number of inputs (2 in this case), and then lists the
  numbers of the decks that are inputs (1 and 2 in this case, or in
  other words \emph{in} and \emph{limit}).

\item The final section holds the actual program instructions. First you
  get the total size of the program (20). Then there are a number of
  lines each corresponding to one instruction with its parameters.
\end{itemize}

\begin{exercise}
  You might expect the numbers of the program instruction lines to be
  ``$1,2,3,\ldots$'', but instead they are ``$0,3,7,\ldots$''. Where do these
  numbers come from?
\end{exercise}

\begin{exercise}
  Use the mnemonic/machine code table above to translate
  the instructions back to readable form, including the
  parameters. Why, for instance, does the first line end with~17?
\end{exercise}

\begin{exercise}
  Note that the instruction list contains \textsc{jump\_less}, but not
  \textsc{jump\_greater}. Is this a mistake? If you can't figure this
  out, try compiling a program that contains a ``\emph{jump if greater}''
  and see what happens with it.
\end{exercise}

It's not too difficult to get oriented in the machine code listing
because of the line breaks and the text and numbers to the left of the
colons. But note that this is just for your convenience. The actual
DeckVM code is just the numbers to the right of the colons:

\begin{lstlisting}
4 2 105 110 5 108 105 109 105 116 3 111 117 116 5 116 114 97 115
104 2 1 2 20 83 1 17 101 1 2 12 33 1 3 71 0 33 1 4 71 0 24 3 11
\end{lstlisting}

Because of the way things are structured in the three sections, it
\emph{is} of course possible to recreate the full structure from just
this.

\begin{exercise}
  Translate the following DeckVM code into human-readable form (by
  hand):
\begin{lstlisting}
2 1 120 5 116 114 97 115 104 1 1 11 83 1 10 24 1 33 1 2 71 0 11
\end{lstlisting}
\end{exercise}

With the programming you have learned on this course, it should
actually be
possible for you to create, in Javascript, a virtual
machine that runs DeckVM code.
%%
\marginpar{If you go into detail, you may need one thing we haven't
  covered: the function \emph{String.fromCharCode}, which creates a
  string from one or more character code values.}
%% 
For the actual deck operations, you could use the stuff from the
\emph{Card Deck Algorithms in Javascript} exercise. The core loop
would be similar to the \emph{fetch-execute cycle} that a
hardware-implemented \textsc{cpu} has, with a \emph{program counter}
to keep track of the current position in the code. You might not find
it the best use of your time to actually execute this project, but
let's at least think about it:

\begin{exercise}
  Roughly sketch the algorithm (or algorithms) that you would use for
  a Javascript function \verb'runDeckJVM(code)', where \verb'code' is
  an array of integers which make up a full Deck Code program in
  DeckVM format.
\end{exercise}


\end{document}
