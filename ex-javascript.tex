\documentclass[a4paper,twoside]{tufte-handout}
\usepackage{listings}
\usepackage{amsmath}
\usepackage{amssymb}
\usepackage{booktabs}
\usepackage[T1]{fontenc}
\usepackage[utf8]{inputenc}
%% \usepackage{graphics}
\usepackage[USenglish]{babel}

\frenchspacing

\newtheorem{exercise}{Exercise}
\newtheorem{example}{Example}
\newtheorem{assignment}{Assignment}

\renewcommand{\theexercise}{\Alph{exercise}}

\newcommand\lbl[1]{\hspace{-1em}\emph{#1:}}

%%\lstset{basicstyle=\ttfamily,backgroundcolor=\color{white},frame=single,rulecolor=\color{gray!20},framesep=10pt,linewidth=12cm}
\lstset{basicstyle=\normalfont\ttfamily\small,frame=single,rulecolor=\color{gray!40}}

\title{Card Deck Algorithms in Javascript}
\author{Jesper Larsson, IT University of Copenhagen}
\date{2014 version}

\begin{document}
\maketitle

%% ===============================================
\section{Getting Started}\label{sec-start}

In this exercise you will try creating objects and calling their
methods, and practice expressing yourself in code. It ties up the bag
for the main coding section of the course by letting you use the \emph{card deck}
objects and algorithms from the first exercises.

Start with the following \textsc{html} (which you can download from
\verb'http://itu.dk/people/jesl/deckjs/use.html' if you
don't want to type it in):
%%
\begin{lstlisting}
<html xmlns="http://www.w3.org/1999/xhtml">
<head>
<title>Card deck code</title>

<script type="text/javascript"
        src="http://itu.dk/people/jesl/deckjs/deck.js">
</script>

</head>

<body>

<script type="text/javascript">
// Your own code goes here.
</script>

</body>
</html>
\end{lstlisting}
%%

As you can see, the first \emph{script} element, the one inside the
\emph{head}, has no content. (The end tag, \verb'</script>' follows
immediately after the start tag.) Instead, it has a \emph{src}
attribute in the start tag, which points out where to get the code. In
this way, you can include code from the outside in a \textsc{html}
document. The effect is the same as if the contents of the file
\emph{deck.js} had been the content of the \emph{script} element.

If you'd like, you can download \emph{deck.js} using the full address
in the \emph{src} attribute, and see what it looks like. But don't
worry if you can't figure out what it all means, you are just going to
use it to create deck objects to use in own code.

The other \emph{script} element, in the \emph{body}, doesn't have any
real code either, just a comment that you should exchange for your own
Javascript code.

Before we go through all the complete deck object interface, let's start with a trivial example program that just creates
a Deck object, reads some cards for it from the user, and then writes
it to the output. This would look something like the following.
%%
\begin{lstlisting}
   var d = deck("myDeck");
   d.read();
   alert(d);
\end{lstlisting}
%%
This corresponds to the following deck code:
%%
\begin{lstlisting}
   deck myDeck input
   output myDeck
   stop
\end{lstlisting}

Replace the comment in the second \emph{script} element in your
\textsc{html} file with this code, and then open it in Firefox to see
that you get the right result.

Also, you can try looking at the code with Firebug. In the
\emph{Console} tab of Firebug, you often see error messages if
something goes wrong, and you can also try setting breakpoints and
stepping through the code.

%% ===============================================
\section{Interface}\label{sec-intf}

The deck interface is defined by the following methods, which
correspond to how decks can be used in deck code.

The available methods are:
%%
\begin{lstlisting}[frame=none]
   parse(inputString)
   moveTopTo(otherDeck)
   moveAllTo(otherDeck)
   compareTop(otherDeck)
   isEmpty()
\end{lstlisting}
%%
Let's go through how to use them by looking at the corresponding
deck code.

%% -----------------------------------------------------------
\subsection{Create and Read}\label{sec-input}

You create a deck object by calling the \emph{deck} function. It takes
one parameter: the name of the deck. So, for instance:
%%
\begin{lstlisting}
  var deckId = deck("deckName");
\end{lstlisting}
%%
This defines a deck with one name to the programmer (\emph{deckId}),
and another to the user (\emph{deckName}). There is no particular
reason not to use the same name for both, but you should be aware that
they have no direct connection.

That created an empty deck. To prompt the user to input a deck, you
first create the deck and then call the \emph{read} method. For instance:
%%
\begin{lstlisting}
  var inDeck = new Deck("in");
  inDeck.read();
\end{lstlisting}
%%
You can \emph{not} have a variable in Javascript with the name
\emph{in}, because \emph{in} is a keyword in Javascript! So,
%%
\verb'var in = new Deck("in")' would fail.

%% -----------------------------------------------------------
\subsection{Movement}\label{sec-move}

The method \emph{moveTopTo} corresponds to \emph{movetop} in deck
code, and \emph{moveAllTo} corresponds to \emph{moveall}. For instance:
%%
\begin{lstlisting}
   inDeck.moveTopTo(tempDeck);
   tempDeck.moveAllTo(outDeck);
\end{lstlisting}
%%
moves the top card from \emph{inDeck} to \emph{tempDeck}, and then puts all
the cards from \emph{tempDeck} on top of \emph{outDeck}.

%% -----------------------------------------------------------
\subsection{Conditions}\label{sec-conds}

To move around in Javascript code, you use constructs like \emph{if} and
\emph{while}. Not \emph{jumps} like in deck code. But the conditions
used for \emph{if} and \emph{while} are similar to the conditions
available for \emph{jump} in deck code.

The method \emph{compareTop} compares the top cards of two decks. If \emph{a} and \emph{b}
are decks, ``a.compareTop(b)'', returns:
%%
\begin{itemize}
\item A value less than zero if the top card on \emph{a} is less than
  the top card on \emph{b}.
\item Zero if the top card on \emph{a} is equal to
  the top card on \emph{b}.
\item A value greater than zero if the top card on \emph{a} is greater than
  the top card on \emph{b}.
\end{itemize}

This is a typical use:
%%
\begin{lstlisting}
  if (inDeck.compareTop(tempDeck) < 0) {
    // Do this if top card on inDeck is < top card on tempDeck.
  }
\end{lstlisting}

Also, you can test if a deck is empty with \emph{isEmpty}, like this
for instance:
%%
\begin{lstlisting}
  while (!inDeck.isEmpty()) {
    // Repeat this for as long as in is not empty.
  }
\end{lstlisting}

%% -----------------------------------------------------------
\subsection{Output}\label{sec-output}

Deck actually also has a \emph{toString} method that is used to
produce a string showing the deck contents. But this is called
implicitly if a deck is used as a string, so you can just do:
%%
\begin{lstlisting}
  alert(deckId);
\end{lstlisting}
%%
Or you can output the deck into the web page, something like this:
%%
\begin{lstlisting}
  document.write("<p>" + deckId + "</p>");
\end{lstlisting}

\clearpage

\section{Examples}\label{sec-examp}

The examples in this section correspond to those of the deck code exercise.

\begin{example}
  Split a deck of cards into two parts whose sizes are as equal as possible.
  \begin{description}
  \item[Input:] One deck of cards named \emph{in}
  \item[Output:] Two decks of cards \emph{out1} and \emph{out2},
    whose numbers of cards differ by at most one.
  \end{description}
\begin{lstlisting}
var inDeck = deck("in");
inDeck.read();
var o1 = deck("out1");
var o2 = deck("out2");
while (!inDeck.isEmpty()) {
    inDeck.moveTopTo(o1);
    if (!inDeck.isEmpty()) {
        inDeck.moveTopTo(o2);
    }
}
document.write("<p>" + o1 + "</p>");
document.write("<p>" + o2 + "</p>");
\end{lstlisting}
\end{example}

\begin{example}\label{extractge}
  Extract all cards greater than or equal to a given value.
  \begin{description}
  \item[Input:] One deck named \emph{limit} containing a single card,
    and one deck named \emph{in} containing any number of cards.
  \item[Output:] One deck named \emph{out}, containing all the cards
    from \emph{in} whose value is no less than the card in the
    \emph{limit} deck.
 \end{description}
\begin{lstlisting}
var inDeck = deck("in");
inDeck.read();
var lim = deck("limit");
lim.read();
var out = deck("out");
var trash = deck("trash");
while (!inDeck.isEmpty()) {
    if (inDeck.compareTop(lim) >= 0) {
        inDeck.moveTopTo(out);
    } else {
        inDeck.moveTopTo(trash);
    }
}
document.write("<p>" + out + "</p>");
\end{lstlisting}
\end{example}

\clearpage

\begin{example}\label{extractsmallest}
  Extract the smallest card from a deck.
  \begin{description}
  \item[Input:] A deck \emph{in}.
  \item[Output:] A deck \emph{min}, containing containing only one
    card from \emph{in}, which has the smallest value of any of the
    cards in \emph{in}.
 \end{description}
\begin{lstlisting}
var inDeck = deck("in");
inDeck.read();
var min = deck("min");
var trash = deck("trash");
while (!inDeck.isEmpty()) {
    if (min.isEmpty()) {
        inDeck.moveTopTo(min);
    } else if (inDeck.compareTop(min) < 0) {
        min.moveTopTo(trash);
        inDeck.moveTopTo(min);
    } else {
        inDeck.moveTopTo(trash);
    }
}
document.write("<p>" + min + "</p>");
\end{lstlisting}
\end{example}

\begin{example}\label{smallbottom}
  Place smallest cards at the bottom.
  \begin{description}
  \item[Input:] A deck \emph{in}.
  \item[Output:] A deck \emph{out}, that contains all the cards from
    in, but with the cards with the smallest value at the
    bottom. (Note that there can be more than one card with this
    value.) The other cards can be in any order.
 \end{description}
\begin{lstlisting}
var inDeck = deck("in");
inDeck.read();
var min = deck("min");
var temp = deck("temp");
var out = deck("out");
while (!inDeck.isEmpty()) {
    if (min.isEmpty() || inDeck.compareTop(min) == 0) {
        inDeck.moveTopTo(min);
    } else if (inDeck.compareTop(min) < 0) {
        min.moveAllTo(temp);
        inDeck.moveTopTo(min);
    } else {
        inDeck.moveTopTo(temp);
    }
}
min.moveAllTo(out);
temp.moveAllTo(out);
document.write("<p>" + out + "</p>");
\end{lstlisting}
\end{example}

\clearpage


%% ===============================================
\section{Exercises}\label{sec-editrun}

The following exercises are the same as in previous exercises, but
this time, you code them in Javascript.

\begin{exercise}
  Split a deck of cards into three parts whose sizes are as equal as possible.
  \begin{description}
  \item[Input:] One deck of cards named \emph{in}
  \item[Output:] Three decks of cards \emph{out1}, \emph{out2}, and
    \emph{out3}, whose numbers of cards differ by at most one.
 \end{description}
\end{exercise}

\begin{exercise}
  Split a deck of cards into two decks. One with cards smaller than a
  given value, and one with cards greater than or equal to that value.
  \begin{description}
  \item[Input:] One deck \emph{limit} containing a single card, and one
    deck \emph{in} containing any number of cards.
  \item[Output:] One deck named \emph{smaller}, containing all the
    cards from \emph{in} whose values are less than the card in the
    \emph{limit} deck, and one deck named \emph{greater} that contains
    the other cards.
  \end{description}
\end{exercise}

\begin{exercise}
  Example\,\ref{smallbottom} above uses two temporary decks \emph{min}
  and \emph{temp}, which are not part of either input or
  output. Change the algorithm so that it only uses one temporary
  deck, \emph{min}, getting rid of \emph{temp}.
\end{exercise}

It's not surprising if you found the previous exercises at least as
difficult to code as the corresponding deck code. But the following,
which is more complex and involves a loop inside a loop, should be
much clearer in Javascript than the spaghetti jumping you had to do in deck
code.

\begin{exercise}
  Place cards in order.
  \begin{description}
  \item[Input:] A deck \emph{in}.
  \item[Output:] A deck \emph{out}, containing all the cards from
    \emph{in} ordered with smaller-value cards below greater-value
    cards, i.e., no card should be on top of any card with a greater
    value.
  \end{description}
\end{exercise}

The same hint applies as in assignment\,1, and you should be able to
use the same algorithm


\end{document}
