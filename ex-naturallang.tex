\documentclass[a4paper,twoside]{tufte-handout}
\usepackage{listings}
\usepackage{amsmath}
\usepackage{amssymb}
\usepackage{booktabs}
\usepackage[T1]{fontenc}
\usepackage[utf8]{inputenc}
%% \usepackage{graphics}
\usepackage[USenglish]{babel}

\frenchspacing

\newtheorem{exercise}{Exercise}
\newtheorem{example}{Example}

\renewcommand{\theexercise}{\Alph{exercise}}

\newcommand\lbl[1]{\hspace{-1em}\emph{#1:}}

\title{Algorithms in Natural Language}
\author{Jesper Larsson, IT University of Copenhagen}
\date{}

\begin{document}
\maketitle

\begin{abstract}
  This text concerns algorithms for reordering decks of cards,
  expressed in natural language. It is used for exercise on the course
  \emph{IT Foundations}, fall 2011, and is part of the syllabus.
\end{abstract}

%% ===============================================
\section{Introduction}\label{sec-intro}

It is a common misconception that knowledge of programming is the same
thing as knowledge of programming languages. Before you can write the
code that solves a task, you must have an idea about the process that
the code should reflect. The description of that process is what we
call an \emph{algorithm}, and this is possible to express in plain
English.

This exercise is about designing algorithms for rearranging decks of
numbered cards, and formulating them in English.

To make it tangible, you should work with an actual deck of
cards. We will use approximately 20~cards, each with a number between
1 and 20 (but any deck of numbered cards will do). Keep the cards on
the table in front of you with the face side up (the one with the
number). You rearrange the cards, moving them around among a
few decks, making all decisions based on what cards you see. When
cards are on top of each other, you can only make decisions based on
the value of the card on top.

It is recommended that you work in pairs so that you can observe each
others card movements and discuss the algorithms. Get help from the
teaching assistants when you need it.

%% ===============================================
\section{Operations}\label{sec-ops}

The input to each algorithm is one or more named decks of cards, the
contents of which may be restricted in some way in the
specification. (For instance, it may be specified that a particular
deck only contains one card.) You can create more decks as part of the
algorithm, and move cards among both the input decks and those you
created yourself. The decks you create are initially empty, so what
you create is really a place for a deck on the table, not the actual
stack of cards. Input decks can be empty, unless they are specified
not to be. Output consists of one or several decks.

We use natural language for the algorithms, but that doesn't
mean that we can use any operation that natural language can
express. There is a strictly defined \emph{interface}: a set of
available operations, as follows:
%% 
\begin{itemize}
\item Creating a new named deck, e.g., ``Create new deck \emph{x}.''
\item Move the top card of a deck on top of another deck, e.g., ``Move
  top card from deck \emph{x} to deck \emph{y}.'' If \emph{x} is empty
  when this operation is reached, it means that there is an error in
  the algorithm.
\item Move all cards of a deck and place them on top of another deck
  without rearranging them, e.g., ``Move all cards from deck \emph{x}
  to deck \emph{y}.'' It is ok for \emph{x} to be empty when this
  operation is reached, it just means that there is no change to any
  of the decks.
\item Jump to another position of the algorithm. We mark positions
  that we can jump to with \emph{labels}. Jumping backwards in the
  algorithm is how we can get \emph{repetition}.
\item Jump conditionally, based on whether the top card of one deck is
  either smaller, greater, or equal to the top card of another. E.g.,
  ``Jump to \emph{startpoint} if top card of deck \emph{x} is greater
  than top card on deck \emph{y}.''
\item Jump conditionally, based on whether a deck is empty or not. E.g.,
  ``Jump to \emph{startpoint} if deck \emph{x} is not empty.''
\item Make a deck \emph{output} of the algorithm, e.g., ``Output deck
  $x$.''
\item Stop. The process is finished.
\end{itemize}

You write one line for each step of an algorithm. You can use pen and
paper, or a computer with a text editor or word processor. Often it is
easier to start on paper and transfer to a computer when it gets too
messy. Each step should be one of the operations listed above,
\emph{nothing else!} In addition to the steps, there are also labels
-- the destination points of jumps. Each label is also written on a
line of its own, followed by a colon.

%% ===============================================
\section{Examples}\label{sec-examp}

Here is a first example of what an algorithm specification can look like:

\begin{example}
  Split a deck of cards into two parts whose sizes are as equal as possible.
  \begin{description}
  \item[Input:] One deck of cards named \emph{in}
  \item[Output:] Two decks of cards \emph{out1} and \emph{out2},
    whose numbers of cards differ by at most one.
 \item[Algorithm:]
  \item\normalfont
    \begin{tabular}{l}
      Create new deck \emph{out1}.\\
      Create new deck \emph{out2}.\\
      \lbl{check}\\
      Jump to \emph{end} if \emph{in} is empty.\\
      Move the top card from \emph{in} to \emph{out1}.\\
      Jump to \emph{end} if \emph{in} is empty.\\
      Move the top card from \emph{in} to \emph{out2}.\\
      Jump to \emph{check}.\\
      \lbl{end}\\
      Output \emph{out1}.\\
      Output \emph{out2}.\\
      Stop.
    \end{tabular}
  \end{description}
\end{example}

Before you read each of the algorithms in the next few examples, try
to come up with a solution yourself. Then read the given algorithm
carefully, make sure you understand how it works (try it for a few
different input cases using the paper cards), and compare with your
own solution ideas. Note that there is always more than one correct
solution, but short and simple are good properties to strive for.

\begin{example}
  Extract all cards greater than or equal to a given value.
  \begin{description}
  \item[Input:] One deck named \emph{limit} containing a single card,
    and one deck named \emph{in} containing any number of cards.
  \item[Output:] One deck named \emph{out}, containing all the cards
    from \emph{in} whose value is no less than the card in the
    \emph{limit} deck.
 \item[Algorithm:]
  \item\normalfont
    \begin{tabular}{l}
      Create new deck \emph{out}.\\
      Create new deck \emph{trash}.\\
      \lbl{check}\\
      Jump to \emph{end} if \emph{in} is empty.\\
      Jump to \emph{skip} if the top card of \emph{in} is smaller than
      the top card of \emph{limit}.\\
      Move the top card from \emph{in} to \emph{out}.\\
      Jump to \emph{check}.\\
      \lbl{skip}\\
      Move the top card from \emph{in} to \emph{trash}.\\
      Jump to \emph{check}.\\
      \lbl{end}\\
      Output \emph{out}.\\
      Stop.
    \end{tabular}
  \end{description}
\end{example}

\begin{example}\label{extractsmallest}
  Extract the smallest card from a deck.
  \begin{description}
  \item[Input:] A deck \emph{in}.
  \item[Output:] A deck \emph{min}, containing containing only one
    card from \emph{in}, which has the smallest value of any of the
    cards in \emph{in}.
 \item[Algorithm:]
  \item\normalfont
    \begin{tabular}{l}
      Create new deck \emph{min}.\\
      Create new deck \emph{trash}.\\
      \lbl{check}\\
      Jump to \emph{end} if \emph{in} is empty.\\
      Jump to \emph{move} if \emph{min} is empty.\\
      Jump to \emph{replace} if top card of \emph{in} is smaller than
      top card of \emph{min}.\\
      Move top card of \emph{in} to \emph{trash}.\\
      Jump to \emph{check}.\\
      \lbl{replace}\\
      Move top card of \emph{min} to \emph{trash}.\\
      \lbl{move}\\
      Move top card of \emph{in} to \emph{min}.\\
      Jump to \emph{check}.\\
      \lbl{end}\\
      Output \emph{min}.\\
      Stop.
    \end{tabular}
  \end{description}
\end{example}

Example\,\ref{extractsmallest} did not specify what should happen if
the input deck is empty. It often happens that some case is missing
from specifications, either because the person who wrote the
specification didn't think of it, or because it doesn't matter what
happens in that case. In the given algorithm, an empty input results
in an empty output, which would seem like a reasonable response.

\begin{example}\label{smallbottom}
  Place smallest cards at the bottom.
  \begin{description}
  \item[Input:] A deck \emph{in}.
  \item[Output:] A deck \emph{out}, that contains all the cards from
    in, but with the cards with the smallest value at the
    bottom. (Note that there can be more than one card with this
    value.) The other cards can be in any order.
 \item[Algorithm:]
  \item\normalfont
    \begin{tabular}{l}
      Create new deck \emph{min}.\\
      Create new deck \emph{temp}.\\
      Create new deck \emph{out}.\\
      \lbl{check}\\
      Jump to \emph{end} if \emph{in} is empty.\\
      Jump to \emph{move} if \emph{min} is empty.\\
      Jump to \emph{move} if top card of \emph{in} is equal to
      top card on \emph{min}.\\
      Jump to \emph{replace} if top card of \emph{in} is smaller than
      top card of \emph{min}.\\
      Move top card of \emph{in} to \emph{temp}.\\
      Jump to \emph{check}.\\
      \lbl{replace}\\
      Move all cards from \emph{min} to \emph{temp}.\\
      \lbl{move}\\
      Move top card of \emph{in} to \emph{min}.\\
      Jump to \emph{check}.\\
      \lbl{end}\\
      Move all cards from \emph{min} to \emph{out}.\\
      Move all cards from \emph{temp} to \emph{out}.\\
      Output \emph{out}.\\
      Stop.
    \end{tabular}
  \end{description}
\end{example}

%% ===============================================
\section{Exercises}\label{sec-exer}

Here follows a number of specifications without given algorithms, to let
find your own solutions. You don't have to hand in anything for this
exercise, but working well now will make it easier for you to solve
the first mandatory assignment.

\begin{exercise}
  Split a deck of cards into three parts whose sizes are as equal as possible.
  \begin{description}
  \item[Input:] One deck of cards named \emph{in}
  \item[Output:] Three decks of cards \emph{out1}, \emph{out2}, and
    \emph{out3}, whose numbers of cards differ by at most one.
 \end{description}
\end{exercise}

\begin{exercise}
  Split a deck of cards into two decks. One with cards smaller than a
  given value, and one with cards greater than or equal to that value.
  \begin{description}
  \item[Input:] One deck \emph{limit} containing a single card, and one
    deck \emph{in} containing any number of cards.
  \item[Output:] One deck named \emph{smaller}, containing all the
    cards from \emph{in} whose values are less than the card in the
    \emph{limit} deck, and one deck named \emph{greater} that contains
    the other cards.
  \end{description}
\end{exercise}

\begin{exercise}
  Example\,\ref{smallbottom} above uses two temporary decks \emph{min}
  and \emph{temp}, which are not part of either input or
  output. Change the algorithm so that it only uses one temporary
  deck, \emph{min}, getting rid of \emph{temp}.
\end{exercise}

\begin{exercise}[Advanced]
  Place cards in order.
  \begin{description}
  \item[Input:] A deck \emph{in}.
  \item[Output:] A deck \emph{out}, containing all the cards from
    \emph{in} ordered with smaller-value cards below greater-value
    cards, i.e., no card should be on top of any card with a greater
    value.
 \end{description}
    
 Start by thinking of of how you would solve this yourself, without
 worrying about formulating an algorithm. Then observe yourself
 ordering the cards, and get your decisions and movements down into an
 algorithm. There are many ways to order cards, but most people
 intuitively come up with one out of two basic algorithm ideas: one
 based on \emph{insertion} and one on \emph{selection}. Even if you
 don't get all the way to writing down the details of the algorithm,
 it is a good exercise to try to come up with ideas for it.
\end{exercise}



\end{document}